\documentclass[a4paper,12pt,english,french]{article}
\usepackage[utf8]{inputenc}
\usepackage[T1]{fontenc}
\usepackage{lmodern}

\usepackage{babel}
\babeltags{eng=english}
\newcommand{\texten}[1]{\texteng{\emph{#1}}}
\frenchbsetup{ItemLabels=\textendash{}, og=«, fg=»}
\usepackage{geometry}
\usepackage[final,babel]{microtype}

\author{Idir Lankri \and{} Yung-Kun Hsieh}
\title{Compte-rendu}
\date{}

\begin{document}
\maketitle{}

\section{Fonctionnalités du programme}
\label{sec:fonct-du-progr}

\paragraph{}

Le programme fourni permet d'afficher trois sortes d'objets
géométriques : des plans (infinis), des sphères et des « boîtes ».  On
peut également appliquer des transformations à ces objets : homothétie,
translation et rotation.

\paragraph{}

Le rendu graphique de la scène décrite dans le fichier scénario se fait
via la bibliothèque \texttt{Graphics}.  Pour avoir une animation, on
affiche simplement une suite d'images en faisant varier la variable
spéciale \texttt{time}.

\paragraph{}

L'utilisateur peut spécifier en ligne de commande la résolution de
l'image, le nombre d'images pour une animation ainsi que le nombre
maximal de rebonds d'un rayon (voir l'option \texttt{-help} du programme
pour plus de détails).

\section{Principaux modules}
\label{sec:principaux-modules}

\subsection{Module \texttt{Scene}}
\label{sec:scene-1}

Ce module se charge de traduire la structure décrivant la syntaxe
abstraite du fichier scénario (type \texttt{Scenario.scenario}) en une
représentation adaptée aux calculs internes du programme (type
\texttt{Scene.t}).  C'est en fait un interpréteur du langage de
description de scène.

\subsection{Module \texttt{Object}}
\label{sec:module-object}

Dans ce module, on définit la représentation des différents objets et la
manière dont les transformations agissent sur ces objets.

\subsection{Module \texttt{Beam}}
\label{sec:module-beam}

C'est dans ce module que l'on fait le \texten{ray tracing}.  On cherche
le premier objet de la scène intersecté par un rayon donné puis on
calcule en conséquence la couleur « vue » par le rayon par récurrence
(par rebonds successifs).

\section{Extensions et améliorations possibles}
\label{sec:extens-et-amel}

\paragraph{}

La principale amélioration possible est de paralléliser le calcul des
images et des pixels étant donné que le calcul de chaque image
(resp. pixel) est indépendant des autres.

\paragraph{}

On pourrait utiliser le \texten{hashconsing} sur les objets
géométriques.  Cela permettrait par exemple d'éviter de reconstruire les
objets fixes de la scène pour chaque image d'une animation.

\paragraph{}

L'utilisateur pourrait choisir la manière dont le rendu graphique est
fait : « en direct » (comme c'est actuellement le cas) ou vers un
fichier image ou vidéo.

\end{document}

%%% Local Variables:
%%% mode: latex
%%% TeX-master: t
%%% End:
